%%%%%%%%%%%%%%%%%%%%%%%%%%%%%%%%%%%%%%%%%
% Lachaise Assignment
% LaTeX Template
% Version 1.0 (26/6/2018)
%
% This template originates from:
% http://www.LaTeXTemplates.com
%
% Authors:
% Marion Lachaise & François Févotte
% Vel (vel@LaTeXTemplates.com)
%
% License:
% CC BY-NC-SA 3.0 (http://creativecommons.org/licenses/by-nc-sa/3.0/)
% 
%%%%%%%%%%%%%%%%%%%%%%%%%%%%%%%%%%%%%%%%%

%----------------------------------------------------------------------------------------
%	PACKAGES AND OTHER DOCUMENT CONFIGURATIONS
%----------------------------------------------------------------------------------------

\documentclass{article}

\input{structure.tex} % Include the file specifying the document structure and custom commands

%----------------------------------------------------------------------------------------
%	INFORMATION
%----------------------------------------------------------------------------------------

\title{Open Source Roboter Plattform} % Title

\author{Lukas Reichwein\\ Yves Ehrlich\\ Nick Gnoevoj}

\date{University of Applied Science Fulda --- \today} % University, school and/or department name(s) and a date

%----------------------------------------------------------------------------------------

\begin{document}

\maketitle % Print the title
\tableofcontents % Inhaltsverzeichniss, Achtung zweimal Compilerien!
\newpage

%----------------------------------------------------------------------------------------
%	INTRODUCTION
%----------------------------------------------------------------------------------------

\section*{Vorwort} % Unnumbered section

	Motivation, Basis, Ziel des Projektes.

\newpage
%----------------------------------------------------------------------------------------
%	Latex Beispeiele
%----------------------------------------------------------------------------------------
\section{Beispiele für Spezielle LaTeX Strukturen}

\begin{info} % Information block
	benutze den Info block um wichtige informationen hervorzuheben.
\end{info}

%----------------------------------------------------------------------------------------
%	Beispiel für Pseudo Code.
%----------------------------------------------------------------------------------------

\begin{center}
	\begin{minipage}{0.5\linewidth} % Adjust the minipage width to accomodate for the length of algorithm lines
		\begin{algorithm}[H]
			\KwIn{$(a, b)$, two floating-point numbers}  % Algorithm inputs
			\KwResult{$(c, d)$, such that $a+b = c + d$} % Algorithm outputs/results
			\medskip
			\If{$\vert b\vert > \vert a\vert$}{
				exchange $a$ and $b$ \;
			}
			$c \leftarrow a + b$ \;
			$z \leftarrow c - a$ \;
			$d \leftarrow b - z$ \;
			{\bf return} $(c,d)$ \;
			\caption{\texttt{FastTwoSum}} % Algorithm name
			\label{alg:fastTwoSum}   % optional label to refer to
		\end{algorithm}
	\end{minipage}
\end{center}

%----------------------------------------------------------------------------------------
%	Beispiel für Code Snippets.
%----------------------------------------------------------------------------------------

% File contents
\begin{file}[hello.py]
\begin{lstlisting}[language=Python]
#! /usr/bin/python

import sys
sys.stdout.write("Hello World!\n")
\end{lstlisting}
\end{file}

%----------------------------------------------------------------------------------------
%	Example for Console Prints (can also be usefull for displaying Serial monitor)
%----------------------------------------------------------------------------------------


% Command-line "screenshot"
\begin{commandline}
	\begin{verbatim}
		$ chmod +x hello.py
		$ ./hello.py

		Hello World!S
	\end{verbatim}
\end{commandline}


% Warning text, with a custom title
\begin{warn}[Notice:]
  Warungen könnten auch nützlich sein, immerhin braucht der RF24 3.3V und nicht 5V
\end{warn}

%----------------------------------------------------------------------------------------
%	Beispiel für ein Bild.
%----------------------------------------------------------------------------------------

\begin{figure}[h]
	\includegraphics[width=8cm]{fish.png}
	\centering
\end{figure}

%----------------------------------------------------------------------------------------
%	BIBLIOGRAPHY
%----------------------------------------------------------------------------------------

\bibliographystyle{unsrt}

\bibliography{sample.bib}

%----------------------------------------------------------------------------------------

\end{document}
